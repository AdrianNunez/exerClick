\section*{Reunión de Trabajo 5}

\textbf{Hora de inicio:} 11:50\\

\textbf{Hora de finalización:} 12:30\\

\textbf{Presentes:} Maite Urretavizcaya, Samara Ruíz, Juan Miguel López, Adrián Núñez\\

\textbf{Temas tratados durante la reunión:}

\begin{itemize}
\item Demostración y discusión del M-2 del UO1-T.
\end{itemize}

\textbf{Resumen de la reunión:}

\begin{itemize}
\item Se empieza mostrando el M-2 del UO1-T desarrollado en un Samsung Galaxy S5 al grupo. La aplicación ha sido añadida al dispositivo mediante USB, no se ha conseguido una apk funcional para instalar la aplicación en cualquier dispositivo.

\item De la próxima reunión en adelante se enviarán capturas de pantalla del modelo para agilizar la reunión y que todos los participantes tengan claro el escenario.

\item Aspectos de diseño a mejorar:
	\begin{itemize}
	\item Los botones rápidos que aparecen en las cajas de los ejercicios deben ir siempre a la derecha, alineados con el 		identificador del ejercicio (aunque haya que acortarlo mucho, se supone que debe ser una clave corta). De esta forma 		la caja ocupa menos. Además, al pulsar el ejercicio para verlo completo (más adelante) se podrá ver el identificador 		entero.
	\item Alguna forma más visual de identificar un ejercicio en el que haya especialmente un número significativo de 			dudas (color de fondo, algo llamativo, texto de otro color, etc.).
	\end{itemize}

\item Se acuerda realizar finalmente el M-3 del UO1-T. Además, se acuerda seguir con el UO1-S (los 3 modelos) para poder realizar así una evaluación más funcional (con el profesor pudiendo lanzar un ejercicio y el alumno respondiendo a este). También se plantea realizar el UO4-T (ver estadísticas de un ejercicio) para poder tener una aplicación verdaderamente funcional.
\end{itemize}

\textbf{Acordado para la siguiente reunión:}

\begin{itemize}
\item M-3 del UO1-T.
\item M-1, M-2 y M-3 del UO1-S y UO4-T.
\item Obtener una apk funcional de la aplicación.
\end{itemize}