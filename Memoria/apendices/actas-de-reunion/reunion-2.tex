\section*{Reunión de Trabajo 2}

\textbf{Fecha:} 21 de octubre de 2014\\

\textbf{Hora de inicio:} 11:00\\

\textbf{Hora de finalización:} 12:25\\

\textbf{Presentes:} Maite Urretavizcaya, Samara Ruiz, Adrián Núñez\\

\textbf{Temas tratados durante la reunión:}

\begin{itemize}
\item Presentación de las ideas pensadas en la fase inicial del proyecto.
\item Partiendo de la discusión de las ideas las decisiones más importantes han sido:
\begin{itemize}
	\item Crear una aplicación para móviles en lugar de una aplicación web.
	\item Iniciar el proyecto teniendo en cuenta que la aplicación sólo va a usarse dentro de clase.
\end{itemize}
\end{itemize}

\textbf{Resumen de la reunión:}

\begin{itemize}
\item Se comienza explicando las ideas reunidas para la aplicación (en el documento de ideas acumuladas por parte del alumno durante las semanas anteriores), de forma que se tenga más claro el tipo de aplicación que se quiere realizar.

\item Samara Ruiz se une a la reunión.

\item Se comentan más ideas y unos bocetos iniciales. Comienza la discusión de la que salen las siguientes ideas:
\begin{itemize}
	\item En lugar de una aplicación web se realizará una aplicación para móviles.
	\item Se limitará de momento el uso de la aplicación a cuando el docente y el alumnado estén en una sesión lectiva 			(excluyendo algunas funciones).
	\item Se deja como pendiente un sistema de valoración de dificultad de los ejercicios por parte del alumno y de 			valoración del grado de satisfacción en la resolución de ejercicios de los alumnos por parte del profesor. Ambas 			usando el sistema que se usa en PresenceClick.
\end{itemize}

\item Se proponen ideas para simplificar los bocetos iniciales. El fin es tener una versión más simple, con menos botones y pantallas (se pretende realizar todo en pocos clicks). Se proponen más ideas respecto a la interfaz y queda pendiente el realizar nuevos bocetos.

\item Se da inicio a la iteración 1 del proyecto como una iteración larga con dos objetivos: responder a un ejercicio (por parte del alumno, UO-1S) y proponer un ejercicio (por parte del profesor, UO-1T). Se escoge como primer UO (Objetivo de Usuario) el de responder a ejercicios por parte del alumno (UO-1S). Los nuevos bocetos serán parte del prototipo en papel para el UO.

\item Se comparte el documento de ideas general con Maite Urretavizcaya y Samara Ruiz.
\end{itemize}

\textbf{Acordado para la siguiente reunión:}

\begin{itemize}
\item Corregir el documento de ideas para el proyecto con lo decidido/propuesto en la reunión.

\item Iniciar la iteración 1 con la fase de Análisis y Captura de requisitos. Enviar cuanto antes unos prototipos en papel resultado de esta fase para ser validados antes de la siguiente reunión.
\end{itemize}
