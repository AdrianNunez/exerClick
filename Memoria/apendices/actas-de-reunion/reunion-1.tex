\section*{Reunión de Trabajo 1}

\textbf{Fecha:} 8 de octubre de 2014\\

\textbf{Hora de inicio:} 12:30\\

\textbf{Hora de finalización:} 13:45\\

\textbf{Presentes:} Maite Urretavizcaya, Adrián Núñez\\

\textbf{Temas tratados durante la reunión:}

\begin{itemize}
\item Inicio del proyecto: presentación del proyecto y de la metodología de trabajo a seguir durante el desarrollo del 	mismo.
\item Acordar tareas a realizar antes de la siguiente reunión.
\end{itemize}

\textbf{Resumen de la reunión:}

\begin{itemize}
\item Se presenta ExerClick: la aplicación web para gestión de ejercicios en el aula. Será una aplicación accesible desde dispositivos pequeños como el teléfono móvil de un alumno hasta dispositivos con pantallas más grandes como las de un ordenador que puede haber en el aula.

\item La primera idea general de ExerClick es que sea una aplicación que puedan manejar tantos alumnos como profesores (dos roles definidos). En principio la idea es que los ejercicios se realicen dentro del aula. Los profesores podrán proponer ejercicios en la aplicación para que los alumnos los realicen. Los alumnos durante o después de la realización del ejercicio podrán responder a la propuesta del profesor: si lo han terminado o han tenido dudas, si están atascados, etc. De esa forma el profesor puede realizar un seguimiento más cercano, rápido y sencillo del alumnado.

\item La filosofía de la aplicación es que tenga mucha funcionalidad pero en pocos Clicks.

\item Se buscan 4 factores fundamentales en la aplicación:
\begin{itemize}
	\item El uso de tecnologías actuales: HTML5, CSS3, PHP, MySQL y Symphony.
	\item La simplicidad de la aplicación (en pocos clicks se deben de poder realizar muchas cosas).
	\item Que sea internacional: internamente estará escrito en inglés (variables, comentarios, etc.) con el fin de que 		pueda llegar a ser código libre accesible a cualquiera. Además se quiere presentar en 4 idiomas (castellano, euskera, 		inglés y francés).
	\item Uso de la tecnología Responsive Web Design (RWD). La aplicación se quiere adaptar a cualquier dispositivo.
\end{itemize}

\item Presentación y explicación general sobre la metodología de desarrollo InterMod a utilizar.
Intermod es una metodología ágil, basada en modelos y centrada en los usuarios.

\item Se identifica el equipo de trabajo:
\begin{itemize}
\item Adrián Núñez, el alumno.
\item Maite Urretavizcaya, la profesora.
\item Juan Miguel López y Begoña Losada, parte del grupo GaLan, que actuarán como validadores.
\item Usuarios finales, tanto profesores como alumnos, que actuarán como validadores.
\end{itemize}
\end{itemize}

\textbf{Acordado para la siguiente reunión:}

\begin{itemize}
\item Inicio de la fase previa al desarrollo de la aplicación: recopilación de información sobre cualquier tema de interés para el proyecto.

\item Generar ideas, prototipos, etc. para tener una visión más concreta del tipo de aplicación que se quiere hacer.
\end{itemize}
