\section*{Reunión de Trabajo 6}

\textbf{Fecha:} 27 de enero de 2015\\

\textbf{Hora de inicio:} 9:30\\

\textbf{Hora de finalización:} 10:10\\

\textbf{Presentes:} Maite Urretavizcaya, Adrián Núñez\\

\textbf{Temas tratados durante la reunión:}

\begin{itemize}
\item Análisis de la validación de los M-2 de UO1-T, UO4-T y UO1-S.
\end{itemize}

\textbf{Resumen de la reunión:}

\begin{itemize}
\item Intercambiar los iconos de ejercicios activos (señal de alerta) y de ejercicios preparados (avión de papel).

\item Modificar los colores de los botones: rojo siempre activo, azul finalizado y amarillo preparado. Por ejemplo, el de lanzar un nuevo ejercicio (que salia con el avión de papel en amarillo) debería cambiar a rojo.

\item Colocar 3 botones por cada ejercicio para poder mover los ejercicios entre los estados activo, finalizado y preparado cuando se desee.

\item Hacer la barra superior más grande y la parte del nombre de la asignatura y el fondo tipo pizarra más pequeño.

\item Usar las fuentes EHU Sans para el idioma castellano y EHU Serif para euskera.

\item A la vista de alumno añadirle la barra extra de progreso de la clase. También añadirle a los ejercicios colores de fondo dependiendo de sus estados.
\end{itemize}

\textbf{Acordado para la siguiente reunión:}

\begin{itemize}
\item Mejorar lo comentado y continuar con el proyecto.

\item Actualizar la lista de UOs con información actualizada.
\end{itemize}