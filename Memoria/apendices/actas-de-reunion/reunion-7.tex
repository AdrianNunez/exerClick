\section*{Reunión de Trabajo 7}

\textbf{Fecha:} 6 de febrero de 2015\\

\textbf{Hora de inicio:} 11:30\\

\textbf{Hora de finalización:} 12:20\\

\textbf{Presentes:} Maite Urretavizcaya, Samara Ruíz, Adrián Núñez\\

\textbf{Temas tratados durante la reunión:}

\begin{itemize}
\item Demostración de los M-3 de los UO1-T, UO3-T, UO4-T y UO1-S.
\end{itemize}

\textbf{Resumen de la reunión:}

\begin{itemize}
\item Se les enseña a los asistentes los M-3 de los UO1-T, UO3-T, UO4-T y UO1-S. La funcionalidad básica (o ciclo principal de la aplicación) de enviar ejercicios y que los alumnos respondan y se refleje esta respuesta en la vista de profesor no está completa.

\item Se deciden realizar las siguientes modificaciones:
\begin{itemize}
	\item Modificar las barras de progreso: colocar un texto más corto, luego la barra y finalmente el porcentaje, 			todo en una misma línea para que no ocupen tanta altura (esto se aplica para las barras de progreso de la vista de profesor y para las de la vista de estudiante).
	
	\item El tamaño de la barra superior y de la zona de título de la asignatura hacerlos más grandes para que se vean mejor.
\end{itemize}

\item La internacionalización de la aplicación (tenerla disponible en 4 idiomas) queda fijada como objetivo a corto plazo para ir pensando y planificando para añadir. Se dan algunas ideas basadas en qClick y PresenceClick del modo de traducir el texto.

\item Se comentan algunas modificaciones para hacer a la memoria (se dejan por escritas en un documento previamente preparado por la directora del proyecto que le es entregado al alumno).
\end{itemize}

\textbf{Acordado para la siguiente reunión:}

\begin{itemize}
\item Seguir intentando generar el archivo .apk.

\item Buscar la internacionalización de la aplicación.

\item Conseguir que el ciclo principal de la aplicación funcione.

\item Corregir la memoria y el diseño de la aplicación con lo comentado durante la reunión.
\end{itemize}