\section*{Reunión de Trabajo 4}

\textbf{Fecha:} 11 de noviembre de 2014\\

\textbf{Hora de inicio:} 11:00\\

\textbf{Hora de finalización:} 11:40\\

\textbf{Presentes:} Maite Urretavizcaya, Samara Ruiz, Adrián Núñez\\

\textbf{Temas tratados durante la reunión:}

\begin{itemize}
\item Planteamiento de la funcionalidad del los UO-1, UO-2, UO-3 y UO-4 mediante los prototipos en papel y los autómatas de estados (fin de M-1).

\item Inicio de M-2 (implementación de la interfaz).
\end{itemize}

\textbf{Resumen de la reunión:}

\begin{itemize}
\item Se muestran los prototipos en papel junto a sus correspondientes autómatas de estados (dos: uno del profesor y otro del alumno).

\item Se acuerda que cada UO tendrá el siguiente contenido:
\begin{itemize}
	\item Nombre del UO.
	\item Descripción breve.
	\item Autómata de estados individual del UO.
	\item Pantallas necesarias para el UO.
\end{itemize}

\item Se visualiza, por cada UO (del 1 al 4), qué pantallas debería usar y cómo debería ser la interacción con el usuario para cumplir el UO. 

\item Se plantean las siguientes ideas/propuestas:
\begin{itemize}
	\item Vista alumno:
	\begin{itemize}
		\item Cuando el alumno responde a un ejercicio, el botón que ha marcado (ya sea que ha acabado o que ha tenido una 		duda) se resaltará. También es posible que se le cambien los bordes al contenedor en el que está el ejercicio.
		\item Siguiendo con los mismos botones: añadir un lapso de tiempo antes de que la respuesta del alumno cuente 				dentro de las estadísticas por si se hubiera equivocado pulsando el botón.
		\item Añadir dos barras de progreso de la sesión: una del alumno y otra de la clase (para poder ver su progreso en 		comparación con el resto de compañeros).
	\end{itemize}
	\item Vista profesor:
	\begin{itemize}
		\item Los iconos para ejercicio acabado o alumno con dudas deben ser iguales para ambas vista: profesor y alumno.
		\item En vez de usar estadísticas del tipo “20/100 han acabado este ejercicio” interesa usar porcentajes (20\%), 			ya que para otros valores puede que no quede tan claro el porcentaje real. 
		\item En la parte de dudas, en estadísticas, se mostrarán sólo los alumnos que tienen en el momento dudas, no los 			que tuvieron anteriormente.
		\item Cuando el ejercicio está finalizado se propone añadir diferentes estadísticas (nuevo UO que queda como 				pendiente): gente que ha tenido duda en algún momento, dudas resueltas, gente que se quedó con dudas, gente que 			terminó el ejercicio y tiempo de realización del mismo.
	\end{itemize}
\end{itemize}

\item Para esta iteración se acuerda que, para simplificar las funcionalidades que vamos a añadir (no queremos añadir todas las funcionalidades de golpe), omitiremos detalles del diseño original que más tarde serán añadidos:
\begin{itemize}
	\item Pestaña de ejercicio preparados.
	\item Opciones avanzados de creación de un ejercicio.
	\item Guardar un ejercicio (para que se envíe a preparados). Sólo se podrá lanzar.
\end{itemize}
Si bien es posible que existan más elementos a omitir, estos no fueron identificados en el momento de la reunión.

\item Se acuerda finalmente iniciar con la implementación de la interfaz mediante las tecnologías HTML y CSS. Además, se usará PhoneGap para portar estas tecnologías a las plataformas móviles (en principio Android, iOS y Windows Phone).
\end{itemize}

\textbf{Acordado para la siguiente reunión:}

\begin{itemize}
\item Implementación de la interfaz en HTML y CSS y adaptación a móviles de esta mediante PhoneGap.
\end{itemize}
