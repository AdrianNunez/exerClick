%----------------------------------------------------%
%                     OBJETIVOS                      %
%----------------------------------------------------%

\pagestyle{fancy}

\chapter{Objetivos del proyecto}
\label{objetivos}

Durante las sesiones prácticas de ejercicios en el aula a menudo el docente se pregunta si todo el mundo ha acabado los ejercicios propuestos con la esperanza de poder continuar con otro ejercicio o alguna explicación. Se puede preguntar si sus alumnos tienen vergüenza de preguntar dudas o si pasan de ello, si no estará consumiendo demasiado tiempo en este ejercicio, etc. Con un conjunto de 100 alumnos, por ejemplo, resulta imposible observar el avance de los alumnos en los ejercicios y atender a todas las dudas que surgen. Incluso la estrategia de dividir la clase en grupos más pequeños para las clases prácticas, que puede parecer una buena idea, acaba consumiendo más tiempo, al tener que repetir lo mismo varias veces.\\

Lo ideal sería que existiera un medio para permitir a los alumnos dejar claro el estado en el que se encuentran, que el profesor pudiera ver como va cada alumno y que si hubiera dudas quedarán en algún sitio almacenadas para no dejar a ningún alumno sin su respuesta.\\

Con estos objetivos nace exerClick, que pretender ayudar a alumnos y docentes a llevar a cabo esta tarea. En este capítulo describiremos el alcance del proyecto.\\

\section{Alcance del proyecto}

Dividiremos los objetivos en objetivos del alumno y objetivos del profesor. Estos son los objetivos fijados para el alumno:

\begin{itemize}
\item Que pueda ver en todo momento cuales son los ejercicios propuestos en clase.
\item Que un alumno pueda dejar constancia del estado en el que se encuentra durante un ejercicio: si lo ha terminado, si tiene dudas o si está realizándolo.
\end{itemize}

Los objetivos del docente serán los siguientes:

\begin{itemize}
\item Que pueda ver en todo momento cuales son los ejercicios que va proponiendo en clase, los que se han terminado durante la sesión presente o los que aún no se han propuesto pero están preparados.
\item Crear en cualquier momento un nuevo ejercicio (rápidamente o bien preparándolo tranquilamente). Y una vez creado tener la posibilidad de proponerlo a la clase o guardarlo para proponerlo más tarde.
\item Ver por cada ejercicio el estado de los alumnos: quiénes lo han terminado, quiénes tienen duda y quiénes no han respondido nada.
\item Dar por finalizado un ejercicio propuesto activo durante la sesión.
\item Poder ver todos los detalles sobre un ejercicio: los detalles han de ser incluidos a mano por el docente (se pretende que estos se preparen fuera de las sesiones prácticas).
\item Editar cualquier detalle de un ejercicio en cualquier momento.
\item Valorar la realización de un ejercicio a un alumno concreto.
\item Poder cambiar de asignatura en cualquier momento.
\end{itemize}

Además, se han fijado dentro del alcance los siguientes requisitos:

\begin{itemize}
\item Contar con la opinión de usuarios finales (alumnos y docentes) durante el desarrollo de la aplicación, asegurando su aceptación general.
\item La internacionalización de la aplicación, que tendrá 4 idiomas disponibles: castellano, euskera, inglés y francés.
\item Desarrollar la aplicación para que funcione en el mayor número de dispositivos posibles: smartphones, tablets, etc., teniendo en cuenta los diferentes tamaños de pantalla.
\item La aplicación será multiplataforma, pudiendo funciona en Android, iOS y Windows Phone.
\end{itemize}

\section{Exclusiones del proyecto}

