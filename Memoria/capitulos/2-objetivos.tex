%----------------------------------------------------%
%                     OBJETIVOS                      %
%----------------------------------------------------%

\pagestyle{fancy}

\chapter{Documento de objetivos del proyecto (DOP)}
\label{objetivos}

Durante las sesiones prácticas de ejercicios en el aula a menudo el docente se pregunta si todo el mundo ha acabado los ejercicios propuestos con la esperanza de poder continuar con otro ejercicio o dando las explicaciones necesarias para poder continuar. Se puede preguntar si sus alumnos tienen vergüenza de plantear dudas o si pasan de ello, si no estará consumiendo demasiado tiempo en este ejercicio, etc. Con un conjunto de 100 alumnos, por ejemplo, resulta imposible observar el avance de los alumnos en los ejercicios y atender a todas las dudas que surgen. Incluso la estrategia de dividir la clase en grupos más pequeños para las clases prácticas, que puede parecer una buena idea, acaba consumiendo más tiempo, al tener que repetir lo mismo varias veces.\\

Lo ideal sería que existiera un medio para permitir a los alumnos dejar claro el estado en el que se encuentran, que el profesor pudiera ver como va cada alumno y que si hubiera dudas quedaran en algún sitio almacenadas para no dejar a ningún alumno sin su respuesta.\\

Con estos objetivos nace exerClick, que pretender ayudar a alumnos y docentes a llevar a cabo esta tarea. En este capítulo describiremos el alcance del proyecto.\\

%----------------------------------------------------%
%                       ALCANCE                      %
%----------------------------------------------------%

\section{Alcance del proyecto}

Dividiremos los objetivos en objetivos del alumno y objetivos del profesor. Estos son los objetivos fijados para el alumno:

\begin{itemize}
\item Que pueda ver en todo momento cuales son los ejercicios propuestos en clase.
\item Que un alumno pueda dejar constancia del estado en el que se encuentra durante un ejercicio: si lo ha terminado, si tiene dudas o si está realizándolo.
\item Ver su progreso y el de sus compañeros en la realización de ejercicios.
\end{itemize}

Los objetivos del docente serán los siguientes:

\begin{itemize}
\item Que pueda ver en todo momento cuales son los ejercicios que va proponiendo en clase, los que se han terminado durante la sesión presente o los que aún no se han propuesto pero están preparados.
\item Crear en cualquier momento un nuevo ejercicio (rápidamente o bien preparándolo tranquilamente). Y una vez creado tener la posibilidad de proponerlo a la clase o guardarlo para proponerlo más tarde.
\item Ver por cada ejercicio el estado de los alumnos: quiénes lo han terminado, quiénes tienen duda y quiénes no han respondido nada.
\item Editar cualquier detalle de un ejercicio en cualquier momento.
\item Valorar la realización de un ejercicio a un alumno concreto.
\end{itemize}

Además, se han fijado dentro del alcance los siguientes requisitos:

\begin{itemize}
\item Contar con la opinión de usuarios finales (alumnos y docentes) durante el desarrollo de la aplicación, asegurando su aceptación general.
\item La internacionalización de la aplicación, que tendrá 4 idiomas disponibles: castellano, euskera, inglés y francés.
\item Desarrollar la aplicación para que funcione en el mayor número de dispositivos posibles: smartphones, tablets, etc., teniendo en cuenta los diferentes tamaños de pantalla.
\item La aplicación será multiplataforma, pudiendo funciona en Android, iOS y Windows Phone.
\end{itemize}


%----------------------------------------------------%
%                     EXCLUSIONES                    %
%----------------------------------------------------%

\section{Exclusiones del proyecto}

\begin{itemize}
\item Se van a excluir del proyecto todas las funciones que se puedan desarrollar en PresenceClick ligadas a exerClick (es decir, sobre ejercicios en el aula, como pueden ser estadísticas más trabajadas).
\end{itemize}


%----------------------------------------------------%
%              FASES Y TAREAS DEL PROYECTO           %
%----------------------------------------------------%

\section{Fases y tareas del proyecto}

\subsection{Estudio de alternativas para la creación de una aplicación móvil}

Al principio el proyecto iba a seguir a su antecesor, \textit{qClick}, como una aplicación web. Sin embargo, acabó pensándose durante una reunión inicial en convertirlo en una aplicación móvil para aprovecharse mejor de las ventajas de un móvil: notificaciones, no requiere necesidad de cargar la página cada vez que se entra, etc.\\

Ya que el proyecto iba a ser una aplicación web y por tanto se iban a usar tecnologías web se acordó usar un sistema como Apache Cordova para la implementación de la aplicación en lugar de crear una aplicación nativa. Así no se requerían de conocimientos de Android y por tanto se mantenían las bases iniciales del proyecto.\\

Se pensó en Apache Cordova ya que era conocido por uno de los integrantes del grupo GaLan, que estaba desarrollando una aplicación con esta tecnología. De esta forma se podía consultar en caso de alguna duda o pregunta en general directamente a dicha persona en lugar de perder excesivo tiempo buscando respuestas.\\

\subsection{Formación}

Ya se partía con una base propia de conocimiento en tecnologías web, por lo que no fue un gran problema.\\

Samara Ruíz fue la que aportó información sobre Responsive Web Design al inicio del proyecto para poder empezar a realizar las interfaces gráficas. No tenía conocimientos al respecto de esta herramienta y fue de gran ayuda una clase rápida introductoria.\\

Para aprender Apache Cordova se siguieron tutoriales encontrados en la red, tanto para crear la primera aplicación como para simulaciones y pruebas. Para algunas cosas también se consultó con el grupo GaLan en ciertos momentos.\\

\subsection{Documentación del proyecto}

La memoria del proyecto se empezó en una etapa temprana del proyecto. Se comenzó creando la plantilla de \LaTeX \ con todo lo necesario desde cero. La plantilla de la universidad resultaba algo incómoda de usar para cambiar algunos detalles, es por eso que se decidió crear una propia. Tras tener el estilo definido y los paquetes necesarios añadidos se realizó la estructura del documento, añadiendo sólo las cabeceras de cada sección, subsección, etc.\\

También al inicio se comenzó a realizar un seguimiento constante. Mediante Google Drive se iban manteniendo hojas de cálculo con cada tarea realizada, su duración y fecha, así como el cómputo total de horas por mes (y dentro de cada mes por iteraciones, si hubiera más de una). De esta forma plasmarlo más adelante fue una tarea muy sencilla.\\

La memoria no se desarrolló demasiado hasta que la aplicación empezaba a verse acabada, momento en el que empezó a tomar cuerpo con mucho contenido. Debido a que en abril se empezaron los estudios en el programa Erasmus en Alemania la memoria quedó completamente parado hasta junio, momento en el que se finalizó exitosamente.\\

\subsection{Presentación y Defensa del proyecto}

La presentación del proyecto se realizará en los meses de agosto y septiembre, momento en el que se dispondrá de todo el tiempo necesario. Se plantea realizar unas transparencias para una presentación de 30 minutos con previo ensayo con la directora del proyecto.\\

%----------------------------------------------------%d
%                 ANALISIS DE RIESGOS                %
%----------------------------------------------------%

\section{Análisis de riesgos}

En este punto se concretan dos riesgos que pueden dar problemas durante el proyecto y la solución encontrada para mitigarlos:\\

\subsection*{Programa Erasmus}

El mayor riesgo durante el proyecto es el cuatrimestre en el programa Erasmus desde el mes de abril al mes de agosto: se desconocen los eventos que puedan acontecer durante esos meses, si la carga lectiva de las asignaturas y trabajos impedirá el seguir con la memoria, o simplemente puede ser una mala época anímicamente hablando para realizar el proyecto. De cualquier modo, para evitar este riesgo se ha pensado en acabar la aplicación por lo menos antes de marchar para minimizar riesgos.\\

\subsection*{Pérdida de información}

La pérdida de información siempre es una posibilidad cuando se trabaja en una aplicación de esta índole. Para tener menos probabilidades de que se pierda absolutamente todo se realizan varias copias de todos los ficheros:

\begin{itemize}
\item Copia principal en el ordenador de trabajo.
\item En Github, en el repositorio exerClick (ver sección \ref{infraestructura}).
\item En Google Drive.
\end{itemize}

Se actualizan periódicamente, sin una frecuencia fija.\\

%----------------------------------------------------%
%              ANALISIS DE FACTIBILIDAD              %
%----------------------------------------------------%

\section{Análisis de factibilidad}

Hay poca carga lectiva durante el primer cuatrimestre (3 asignaturas), dando tiempo a desarrollar el proyecto. Aun así, se ha visto que antes de empezar el programa Erasmus hay 3 meses sin ningún tipo de carga a parte del Trabajo de Fin de Grado. Sumado a que debido al Erasmus sólo se puede defender el proyecto en Septiembre y se tiene tiempo en verano para continuar tenemos una gran cantidad de tiempo prevista. Esto hace que la realización del proyecto sea completamente factible.\\