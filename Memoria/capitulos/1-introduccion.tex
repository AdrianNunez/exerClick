%----------------------------------------------------%
%                    INTRODUCCION                    %
%----------------------------------------------------%

\pagestyle{fancy}

\chapter{Introducción}
\label{introduccion}

\section{Contexto}

Muchas veces nos encontramos con aulas con demasiados alumnos. Estas clases son especialmente comunes en los primeros cursos de los estudios universitarios, donde el número de alumnos alcanza fácilmente las tres cifras. Responder a todas las preguntas y dudas, monitorizar a los alumnos, asegurar que todos o la mayoría hayan acabado el ejercicio, etc. con esa cantidad de alumnos no es posible para una sola persona. Ante estas situaciones la mayoría de veces suelen ignorarse problemas y dudas y seguir adelante.\\

Los nuevos planes de estudio que pretenden dejar atrás el sistema de educación mediante clases magistrales y dinamizar las clases han supuesto un aumento en el número de clases prácticas y laboratorios que se realizan. Algunos centros han optado incluso por dividir las clases en grupos más pequeños para realizar las prácticas, pero muchas veces ésto no es posible. En esas situaciones el profesor acaba por no poder monitorizar completamente la clase.\\

Con el fin de tener un medio común se han implantado en los últimos años nuevas tecnologías en entornos docentes. Sin embargo, en muchos casos esta tecnología se limita a entornos de apoyo a la docencia más que al alumnado, siendo muy popular el sistema de gestión del aprendizaje Moodle. Además, el uso más frecuente de estos sistemas es el de simple almacén de recursos bibliográficos (enlaces, apuntes, transparencias, etc.).\\

Por otro lado, la expansión de las tecnologías móviles y tabletas, con las que los alumnos están cada vez más familiarizados, no ha sido aprovechada. Estas tecnologías están ya mayoritariamente presentes en las aulas, la mayoría del alumnado dispone de alguno de estos dispositivos, pero su uso como herramienta educativa no es real, desperdiciando así todo su potencial como sistema de ayuda al aprendizaje. Es más, muchas veces el uso de estos dispositivos está prohibido o limitado en clase.\\

\section{Propuesta}

Nuestra propuesta pretende modificar y actualizar los modelos educativos presenciales a través de herramientas que faciliten la captura de la información de lo que sucede en el aula, con el objetivo de proporcionar \textit{feedback} tanto a profesores y estudiantes sobre su progreso en el aprendizaje. Esta propuesta se materializa en la aplicación PresenceClick que facilita la captura colaborativa de esta información entre alumnos y profesores de manera ágil. PresenceClick actualmente dispone de una serie de módulos que capturan la asistencia de los alumnos a clase de manera automática, sus sensaciones sobre las diversas actividades de aprendizaje, sus respuestas a preguntas al aire del profesor y sus dudas. En particular, nuestro objetivo en este nuevo proyecto es capturar las interacciones entre profesor-alumnos en sesiones de ejercicios; cuando el profesor establece los ejercicios a realizar y los alumnos avisan de su realización.\\

El profesor dispondrá de una interfaz web a la que accederá mediante portátil o tableta en clase desde donde indicará a sus alumnos los ejercicios a realizar (está actividad se podrá realizar previa a la clase). Inicialmente se podrá indicar una planificación de todos los posibles ejercicios, con el fin de realizar un mejor seguimiento de los mismos. Se podrá disponer en tiempo real de información sobre el porcentaje de alumnos que lo han realizado, alumnos que indican problemas en su realización y aquellos alumnos que no indican nada. De esta forma el profesor tendrá idea en tiempo real de quiénes y cuántos han realizado el ejercicio y podrá acercarse a comprobar y revisar sus soluciones. Además el profesor podrá valorar su nivel de corrección o satisfacción en la realización, añadiendo las notas oportunas en el sistema que le permitirá seguir la evolución de cada uno de sus alumnos durante el curso. También podrá acercarse a aquellos que señalan problemas en su realización, con el fin de ayudarlos y evitar dificultades en su progreso.\\

Bajo este contexto surge \textbf{exerClick}, la herramienta para seguimiento de ejercicios en el aula. Esta herramienta, con todas sus funcionalidades, nace con el propósito de tener una visión más real de los alumnos, tanto en grupo como individualmente y está dirigida a profesores y a los propios alumnos. De esta manera, el docente puede ofrecer un aprendizaje más adaptado e individualizado, aunque los grupos de alumnos sean muy grandes.\\

\section{Organización del documento}

En este documento se describe principalmente la gestión y el seguimiento realizado durante el desarrollo de \textbf{exerClick}.\\

En el capítulo 1 se ha introducido el contexto en el que nos situamos y se ha explicado brevemente nuestra propuesta. En el capítulo 2 se describe la motivación para desarrollar la aplicación y los objetivos a alcanzar. En el capítulo 3 se detalla la gestión llevada a cabo.\\