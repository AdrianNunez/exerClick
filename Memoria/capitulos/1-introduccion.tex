%----------------------------------------------------%
%                    INTRODUCCION                    %
%----------------------------------------------------%

\pagestyle{fancy}

\chapter{Introducción}
\label{introduccion}

El uso de las tecnologías en entornos docentes ha sido creciente en los últimos años. Sin embargo, en muchos casos esta tecnología se limita a entornos de apoyo a la docencia, siendo muy popular el sistema de gestión del aprendizaje Moodle. Además, el uso más frecuente de estos sistemas es el de simple almacén de recursos bibliográficos (enlaces, apuntes, trasparencias, etc.). Por otro lado, la expansión de las tecnologías móviles, con las que los alumnos están cada vez mas familiarizados, no ha sido aprovechada. Estas tecnologías están ya mayoritariamente presentes en las aulas, pero su uso como herramienta educativa no es real, desperdiciando así todo su potencial como sistema de ayuda al aprendizaje.\\

Nuestra propuesta pretende modificar y actualizar los modelos educativos presenciales a través de herramientas que faciliten la captura de la información de lo que sucede en el aula, con el objetivo de proporcionar feedback tanto a profesores y estudiantes sobre su progreso en el aprendizaje. Esta propuesta se materializa en la aplicación PresenceClick que facilita la captura colaborativa de esta información entre alumnos y profesores de manera ágil. PresenceClick actualmente dispone de una serie de módulos que capturan la asistencia de los alumnos a clase de manera automática, sus sensaciones sobre las diversas actividades de aprendizaje, sus respuestas a preguntas al aire del profesor y sus dudas. En particular, nuestro objetivo en este nuevo proyecto es capturar las interacciones entre profesor-alumnos en sesiones de ejercicios; cuando el profesor establece los ejercicios a realizar y los alumnos avisan de su realización.\\

El profesor dispondrá de una interfaz web a la que accederá mediante portátil o tableta en clase
desde donde indicará a sus alumnos los ejercicios a realizar (está actividad se podrá realizar previa
a la clase). Inicialmente se podrá indicar una planificación de todos los posibles ejercicios, con el
fin de realizar un mejor seguimiento de los mismos. Se podrá disponer en tiempo real de
información sobre el porcentaje de alumnos que lo han realizado, alumnos que indican problemas
en su realización y aquellos alumnos que no indican nada. De esta forma el profesor tendrá idea
en tiempo real de quiénes y cuántos han realizado el ejercicio y podrá acercarse a comprobar y
revisar sus soluciones. Además el profesor podrá valorar su nivel de corrección o satisfacción en la
realización, añadiendo las notas oportunas en el sistema que le permitirá seguir la evolución de
cada uno de sus alumnos durante el curso. También podrá acercarse a aquellos que señalan
problemas en su realización, con el fin de ayudarlos y evitar dificultades en su progreso.
Esta herramienta, con todas sus funcionalidades, nace con el propósito de tener una visión más
real de los alumnos, tanto en grupo como individualmente y está dirigida a profesores y a los propios alumnos. De esta manera, el docente puede ofrecer un aprendizaje más adaptado e
individualizado, aunque los grupos de alumnos sean muy grandes.\\