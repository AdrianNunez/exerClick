%----------------------------------------------------%
%              DISEÑO E IMPLEMENTACIÓN               %
%----------------------------------------------------%

\chapter{Diseño e Implementación}
\label{diseno-e-implementacion}

Añadir descripcion del apartado.\\

\section{Estructura del proyecto}
\label{diseno-e-implementacion:estructura}

Hablar sobre la estructura de ficheros del proyecto, sobre la estructura que sigue Cordova en los ficheros y sobre como está todo en eclipse.\\

\section{Interfaces o lado del cliente}
\label{diseno-e-implementacion:interfaces}

Hablar sobre la parte del cliente, las interfaces y el javascript.\\

\section{Lógica de negocio o lado del servidor}
\label{diseno-e-implementacion:logica-negocio}

Hablar sobre la parte de servidor y la base de datos.\\

\subsection{Base de datos}
\label{diseno-e-implementacion:logica-negocio:bd}

El diseño de la base de datos se ha realizado considerando desarrollos anteriores del grupo GaLan.
Por ello muchos de los elementos son comunes con una de sus herramientas genérica de creación
de sistemas docente, MAGADI, que ha sido proporcionada para este proyecto.\\

Se han incorporado unas tablas nuevas a la base de datos ya existente para la gestión de ejercicios.\\