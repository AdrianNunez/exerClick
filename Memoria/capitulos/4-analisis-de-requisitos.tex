%----------------------------------------------------%
%              ANALISIS DE REQUISITOS                %
%----------------------------------------------------%

\chapter{Análisis de Requisitos}
\label{analisis-de-requisitos}

\section{Requisitos no-funcionales}
\label{analisis-de-requisitos:no-funcionales}

La aplicación ha sido desarrollada con dos requisitos básicos:

\begin{itemize}
\item Que tenga mucha funcionalidad con el mínimo posible de \textit{clicks}.
\item Que para el alumno no suponga una distracción.
\end{itemize}

Asi pues se ha creado un diseño agradable que pretende tener mucha funcionalidad, especialmente en la vista de profesor. En la vista del alumno se ha minimizado el diseño y los elementos para que no existan distracciones importantes en clase.\\

\subsection{Requisitos de la interfaz}
\label{analisis-de-requisitos:no-funcionales:interfaz}

Para hacer más intuitiva la aplicación se han concretado unos colores e iconos específicos para cada tipo de ejercicio, de modo que sea fácil asociar una acción o estado mediante un color o un icono.

\begin{itemize}
\item Ejercicios activos: color rojo, pretendiendo indicar actividad, y un icono de un avión de papel, simbolizando un ejercicio que ha sido lanzado.
\item Ejercicios finalizados: color azul, indicando calma (ya hemos terminado), y un icono de una bandera de meta, simbolizando el fin.
\item Ejercicios preparados: color amarillo y un icono de un reloj, indicando que esto no se ha lanzado aun (viendo las aplicaciones móvil más utilizadas resulta intuitivo el significado del reloj).
\end{itemize}

Para las estadísticas se ha dejado el color verde y un icono de un gráfico de barras. El color azul se emplea para otros botones, sin usar ningún icono.\\

\section{Requisitos funcionales}
\label{analisis-de-requisitos:funcionales}

En este apartado se presentan los requisitos funcionales recogidos en interfaces de papel

Siguiendo las pautas fijadas por los requisitos no-funcionales se han realizado interfaces sencillas e intuitivas. Se le ha dado mucha importancia a la filosofia de "en pocos \textit{clicks}" que sigue la aplicación. Por tanto, ante todo, se han minimizado la cantidad de transiciones entre pantallas y el uso de excesivos botones para buscar mucha funcionalidad en pocos \textit{clicks}.\\

El resultado queda plasmado en estos modelos de requisitos (M-1) de cada objetivo de usuario (User Objective).\\

\subsection{UO1-S: Responder a un ejercicio}
\label{analisis-de-requisitos:funcionales:uo1s}

Añadir pantallas, autómata de estados, info de seguimiento, etc.\\

\subsection{UO1-T: Crear-Lanzar un ejercicio simple}
\label{analisis-de-requisitos:funcionales:uo1t}

Añadir pantallas, autómata de estados, info de seguimiento, etc.\\

\subsection{UO2-T: Crear-Lanzar un ejercicio detallado}
\label{analisis-de-requisitos:funcionales:uo2t}

Añadir pantallas, autómata de estados, info de seguimiento, etc.\\

\subsection{UO3-T: Cambiar el tipo de ejercicio}
\label{analisis-de-requisitos:funcionales:uo3t}

Añadir pantallas, autómata de estados, info de seguimiento, etc.\\

\subsection{UO4-T: Ver estadísticas de un ejercicio}
\label{analisis-de-requisitos:funcionales:uo4t}

Añadir pantallas, autómata de estados, info de seguimiento, etc.\\

\subsection{UO5-T: Ver la descripción completa de un ejercicio}
\label{analisis-de-requisitos:funcionales:uo5t}

Añadir pantallas, autómata de estados, info de seguimiento, etc.\\

\subsection{UO6-T: Editar un ejercicio}
\label{analisis-de-requisitos:funcionales:uo6t}

Añadir pantallas, autómata de estados, info de seguimiento, etc.\\

\subsection{UO7-T: Cerrar sesión}
\label{analisis-de-requisitos:funcionales:uo7t}

Añadir pantallas, autómata de estados, info de seguimiento, etc.\\

\subsection{UO8-T: Cambiar el idioma de la aplicación}
\label{analisis-de-requisitos:funcionales:uo8t}

Añadir pantallas, autómata de estados, info de seguimiento, etc.\\

\subsection{UO9-T: Cambiar de asignatura}
\label{analisis-de-requisitos:funcionales:uo9t}

Añadir pantallas, autómata de estados, info de seguimiento, etc.\\

\section{Requisitos de dispositivos para su ejecución}
\label{analisis-de-requisitos:dispositivos}

La aplicación está pensada para el uso en cualquier dispositivo móvil con los sistemas operativos Android e iOS. No está pensada para tamaños de pantalla excesivamente pequeños, donde probablemente la aplicación se vería incorrectamente.\\

En Android la versión mínima requerida es la 4.0 (API 15).\\