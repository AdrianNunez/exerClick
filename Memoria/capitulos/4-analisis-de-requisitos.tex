%----------------------------------------------------%
%              ANALISIS DE REQUISITOS                %
%----------------------------------------------------%

\chapter{Análisis de Requisitos}
\label{analisis-de-requisitos}

\section{Requisitos no-funcionales}
\label{analisis-de-requisitos:no-funcionales}



\section{Requisitos funcionales}
\label{analisis-de-requisitos:funcionales}

En este apartado se presentan los requisitos funcionales recogidos en interfaces de papel

Siguiendo las pautas fijadas por los requisitos no-funcionales se han realizado interfaces sencillas e intuitivas. Se le ha dado mucha importancia a la filosofia de "en pocos \textit{clicks}" que sigue la aplicación. Por tanto, ante todo, se han minimizado la cantidad de transiciones entre pantallas y el uso de excesivos botones para buscar mucha funcionalidad en pocos \textit{clicks}.\\

\subsection{UO1-S: Responder a un ejercicio}
\label{analisis-de-requisitos:funcionales:uo1s}

Añadir pantallas, autómata de estados, info de seguimiento, etc.\\

\subsection{UO1-T: Crear-Lanzar un ejercicio simple}
\label{analisis-de-requisitos:funcionales:uo1t}

Añadir pantallas, autómata de estados, info de seguimiento, etc.\\

\subsection{UO2-T: Crear un ejercicio detallado}
\label{analisis-de-requisitos:funcionales:uo2t}

Añadir pantallas, autómata de estados, info de seguimiento, etc.\\

\subsection{UO3-T: Cambiar el tipo de ejercicio}
\label{analisis-de-requisitos:funcionales:uo3t}

Añadir pantallas, autómata de estados, info de seguimiento, etc.\\

\subsection{UO4-T: Ver estadísticas de un ejercicio}
\label{analisis-de-requisitos:funcionales:uo4t}

Añadir pantallas, autómata de estados, info de seguimiento, etc.\\

\subsection{UO5-T: Ver la descripción completa de un ejercicio}
\label{analisis-de-requisitos:funcionales:uo5t}

Añadir pantallas, autómata de estados, info de seguimiento, etc.\\

\subsection{UO6-T: Editar un ejercicio}
\label{analisis-de-requisitos:funcionales:uo6t}

Añadir pantallas, autómata de estados, info de seguimiento, etc.\\

\subsection{UO7-T: Cerrar sesión}
\label{analisis-de-requisitos:funcionales:uo7t}

Añadir pantallas, autómata de estados, info de seguimiento, etc.\\

\subsection{UO8-T: Cambiar el idioma de la aplicación}
\label{analisis-de-requisitos:funcionales:uo8t}

Añadir pantallas, autómata de estados, info de seguimiento, etc.\\

\subsection{UO9-T: Cambiar de asignatura}
\label{analisis-de-requisitos:funcionales:uo9t}

Añadir pantallas, autómata de estados, info de seguimiento, etc.\\